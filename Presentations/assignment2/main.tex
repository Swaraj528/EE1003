\documentclass{beamer}
\mode<presentation>
\usepackage{amsmath}
\usepackage{amssymb}
%\usepackage{advdate}
\usepackage{adjustbox}
\usepackage{subcaption}
\usepackage{enumitem}
\usepackage{multicol}
\usepackage{mathtools}
\usepackage{listings}
\usepackage{url}
\def\UrlBreaks{\do\/\do-}
\usetheme{Madrid}
\usecolortheme{lily}
\setbeamertemplate{footline}
{
	\leavevmode%
	\hbox{%
		\begin{beamercolorbox}[wd=\paperwidth,ht=2.25ex,dp=1ex,right]{author in head/foot}%
			\insertframenumber{} / \inserttotalframenumber\hspace*{2ex} 
		\end{beamercolorbox}}%
		\vskip0pt%
	}
	\setbeamertemplate{navigation symbols}{}

	\providecommand{\nCr}[2]{\,^{#1}C_{#2}} % nCr
	\providecommand{\nPr}[2]{\,^{#1}P_{#2}} % nPr
	\providecommand{\mbf}{\mathbf}
	\providecommand{\pr}[1]{\ensuremath{\Pr\left(#1\right)}}
	\providecommand{\qfunc}[1]{\ensuremath{Q\left(#1\right)}}
	\providecommand{\sbrak}[1]{\ensuremath{{}\left[#1\right]}}
	\providecommand{\lsbrak}[1]{\ensuremath{{}\left[#1\right.}}
	\providecommand{\rsbrak}[1]{\ensuremath{{}\left.#1\right]}}
	\providecommand{\brak}[1]{\ensuremath{\left(#1\right)}}
	\providecommand{\lbrak}[1]{\ensuremath{\left(#1\right.}}
	\providecommand{\rbrak}[1]{\ensuremath{\left.#1\right)}}
	\providecommand{\cbrak}[1]{\ensuremath{\left\{#1\right\}}}
	\providecommand{\lcbrak}[1]{\ensuremath{\left\{#1\right.}}
	\providecommand{\rcbrak}[1]{\ensuremath{\left.#1\right\}}}
	\theoremstyle{remark}
	\newtheorem{rem}{Remark}
	\newcommand{\sgn}{\mathop{\mathrm{sgn}}}
	\providecommand{\abs}[1]{\left\vert#1\right\vert}
	\providecommand{\res}[1]{\Res\displaylimits_{#1}} 
	\providecommand{\norm}[1]{\lVert#1\rVert}
	\providecommand{\mtx}[1]{\mathbf{#1}}
	\providecommand{\mean}[1]{E\left[ #1 \right]}
	\providecommand{\fourier}{\overset{\mathcal{F}}{ \rightleftharpoons}}
	%\providecommand{\hilbert}{\overset{\mathcal{H}}{ \rightleftharpoons}}
	\providecommand{\system}{\overset{\mathcal{H}}{ \longleftrightarrow}}
	%\newcommand{\solution}[2]{\textbf{Solution:}{#1}}
	%\newcommand{\solution}{\noindent \textbf{Solution: }}
	\providecommand{\dec}[2]{\ensuremath{\overset{#1}{\underset{#2}{\gtrless}}}}
	\newcommand{\myvec}[1]{\ensuremath{\begin{pmatrix}#1\end{pmatrix}}}
		\let\vec\mathbf

		\lstset{
			%language=C,
			frame=single, 
			breaklines=true,
			columns=fullflexible
		}

		\numberwithin{equation}{section}

		\title{12.8.1.7}
		\author{EE24BTECH11018 - Durgi Swaraj Sharma}
		\date{}
		\begin{document}
		\frame{\titlepage}

		\begin{frame}
			\textbf{Exercise 8.1 Q.7} Find the area of the smaller part of the circle $x^2+y^2=a^2$ cut off by the line $x=\frac{a}{\sqrt{2}}$.\\
		\end{frame}
		\begin{frame}
			\frametitle{\textbf{Theroretical solution}}
			\begin{align}
				x^2+y^2=a^2\\
				y^2=a^2-x^2
			\end{align}
			Taking square root on both sides,
			\begin{align}
				y = \pm\sqrt{a^2-x^2}\\
			\end{align}
			Exploiting symmetry of our problem along the $x$-axis,
			\begin{align}
				y = \sqrt{a^2-x^2} \label{eq1}
			\end{align}
			The smaller part of the circle cut by $x=\frac{a}{\sqrt{2}}$ is the region between $x=\frac{a}{\sqrt{2}} and x = a$. So we find the area of this region as follows.
			\begin{align}
				\text{New area}\, &= \cdot\int\limits_{\frac{a}{\sqrt{2}}}^{a} \sqrt{a^2-x^2}
			\end{align}
		\end{frame}
		\begin{frame}
			Changing to polar coordinates,
			\begin{align}
				x = a \cos\theta\\
				dx = - a \sin\theta d\theta\\
				\text{Plugging into our integral,}\, &\int\limits_{\frac{\pi}{4}}^{0} \sqrt{a^2-a^2 \cos^2\theta} \brak{-a \sin\theta} d\theta\\
				&= \int\limits_{\frac{\pi}{4}}^{0} \sqrt{a^2 \sin^2\theta} \brak{-a \sin\theta} d\theta\\
				&= \int\limits_{\frac{\pi}{4}}^{0} a \sin\theta \brak{-a \sin\theta} d\theta\\
				&= -\int\limits_{\frac{\pi}{4}}^{0} a^2 \sin^2\theta d\theta
			\end{align}
		\end{frame}
		\begin{frame}
			\begin{align}
				&= -a^2\sbrak{\frac{x}{2}-\frac{\sin2x}{4}}_{\frac{\pi}{4}}^{0}\\
				&= -a^2\sbrak{0 -\frac{\pi}{8} - 0 + \frac{1}{4}}
			\end{align}
			\begin{align}
				\text{Required area}\, &= 2\cdot\text{New area} \, = 2\cdot a^2\cdot\frac{\pi-2}{8} = a^2\cdot\frac{\pi-2}{4}
			\end{align}
			We have theoretically found the area of the smaller part of the circle cut by the line to be $a^2 \frac{\pi-2}{4}$.\\
		\end{frame}
		\begin{frame}
			\frametitle{\textbf{Computational Solution}}
			To find the desired area computationally, we'll be utilising the Trapezoidal Rule.\\
			The Trapezoidal rule works by approximating the region under the graph of a function as trapezoids and calculating their area.
			\begin{align}
				\int\limits_a^b f\brak{x}dx &\approx \frac{1}{2}\brak{f\brak{a}+f\brak{b}}\cdot\brak{b-a}\\
				\int\limits_a^b f\brak{x}dx &\approx \sum\limits_{k=1}^{N}\frac{f\brak{x_{k-1}}+f\brak{x_k}}{2}\cdot\brak{x_{k}-x_{k-1}}
			\end{align}
		\end{frame}
		\begin{frame}
			Following from \ref{eq1}
			\begin{align}
				y &= \sqrt{a^2-x^2} \nonumber\\
				\int\limits_{x_n}^{x_{n+1}} y \, dx &= \int\limits_{x_n}^{x_{n+1}} \sqrt{a^2-x^2} dx
			\end{align}
			We can solve the integral on the R.H.S. using the Trapezoidal Rule as follows.
			\begin{align}
				A_{n+1}-A_{n} = 	\int\limits_{x_n}^{x_{n+1}} y \, dx &= h\sbrak{\frac{\sqrt{a^2-x_{n+1}^2} + \sqrt{a^2-x_n^2}}{2}}
			\end{align}
			Where $n$ is the number of iterations we want to calculate in, $h = \frac{a-\frac{a}{\sqrt{2}}}{n}$, $A_{n}$ is the area calculated till the $n^{th}$ iteration, and $x_0 = \frac{a}{\sqrt{2}}$.
		\end{frame}
		\begin{frame}
			The update equation for our area will be:
			\begin{align}
				A_{n+1} = A_{n} +  h\sbrak{\frac{\sqrt{a^2-x_{n+1}^2} + \sqrt{a^2-x_n^2}}{2}}
			\end{align}
			The smaller our step-size, $h$, is, the more accurate our area calulation will be.\\
			And the required area is twice of the calculated area, as the calculated region reflects about the $x$-axis in the total region.
			\begin{align*}
				\text{Area calulcated when}\, a=13 \,\text{from:}\\
				\text{Theoretical Solution is}\, 48.232289\\
				\text{Computational Solution is}\, 48.232288
			\end{align*}
		\end{frame}
	\end{document}
